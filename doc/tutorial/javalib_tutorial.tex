%%%%%%%%%%%%%%%%%%%%%%%%%%%%%%%%
% GENERATED FILE, DO NOT MODIFY.
%%%%%%%%%%%%%%%%%%%%%%%%%%%%%%%%

\documentclass{article}
\usepackage{amsmath}
\usepackage[mathletters]{ucs}
\usepackage[utf8x]{inputenc}
\usepackage{listings}

\title{Javalib Tutorial}
\author{Nicolas Barré}

\begin{document}

\maketitle
\newpage

\tableofcontents
\newpage
\section{Introduction}

\emph{Javalib} is a library written in \emph{OCaml} with the aim to
provide a high level representation of \emph{Java} \textbf{.class}
files. Thus it stands for a good starting point for people who want
to develop static analysis for \emph{Java} byte-code programs,
benefiting from the strength of \emph{OCaml} language.

This document briefly presents the API provided by \emph{Javalib}
and gives some samples of code in \emph{OCaml}.

\section{Global architecture}

All modules of \emph{Javalib} are sub-modules of the package module
\emph{Javalib\_pack} in order to avoid possible namespace
conflicts. The user interface consists of three modules,
\emph{JBasics}, \emph{JCode} and \emph{Javalib}. These modules are
the only ones someone should need to write static analysis. The
other ones provide lower level functions and should not be used.

\subsection{\emph{JBasics} module}

This module gives a representation of all the basic types that
appear at the byte-code level.

These types are:

\begin{itemize}
\item
  JVM very basic types (such as int, float, double, \ldots{})
\item
  JVM more elaborated types (arrays, objects)
\item
  class names, method signatures and field signatures
\item
  others: constant pool types and stackmaps types
\end{itemize}
The types representing class names, method signatures and field
signatures are abstract. The directives to build them are in
\emph{JBasics} and are respectively \textbf{make\_cn},
\textbf{make\_ms} and \textbf{make\_fs}.

This module also provides some sets and maps containers relative to
the abstract class names, field signatures and method signatures.
These maps and sets are very efficient because they rely on the
hidden indexation of each abstract type, thereby reducing the cost
of comparison operations between types. Moreover the maps are based
on \emph{Patricia Trees} implementation, that is known to be really
fast.

\subsection{\emph{JCode} module}

This module provides a representation of the JVM opcodes and of the
code structure at the byte-code level.

It is important to notice that the code is represented by an array
of opcodes, and that the position of an opcode in this array
corresponds to its relative address in the original binary file
(this is the same numbering as \textbf{javap}). In \emph{Javalib}
representation, an opcode includes an instruction and its
arguments. That's why we needed to introduce a dummy opcode
\textbf{OpInvalid} to keep the correct numbering.

\subsection{\emph{Javalib} module}

This is the main module of \emph{Javalib} which contains the
definition of classes, fields and methods types, some methods to
get a high level representation from binary classes and an internal
module \emph{JPrint} to pretty-print or dump every type defined in
the three user modules.

\section{Tutorial}

To begin this tutorial, open an \emph{OCaml} toplevel, for instance
using the \emph{Emacs} \textbf{tuareg-mode}, and load the following
libraries in the given order:

\begin{verbatim}
    #load "str.cma"
    #load "unix.cma"
    #load "extLib.cma"
    #load "zip.cma"
    #load "ptrees.cma"
    #load "javalib.cma"
\end{verbatim}
Don't forget the associated \textbf{\#directory} directives that
allow you to specify the paths where to find these libraries. If
you installed javalib with findlib you should do:

\begin{verbatim}
    #directory "<package_install_path>extlib"
    #directory "<package_install_path>camlzip"
    #directory "<package_install_path>ptrees"
    #directory "<package_install_path>javalib"
    (*<package_install_path> is given by command 'ocamlfind printconf'. 
    If it is the same path than standard ocaml library just replace by '+'.*)
\end{verbatim}
You can also build a toplevel including all these libraries using
the command \textbf{make ocaml} in the sources repository of
\emph{Javalib}. This command builds an executable named
\textbf{ocaml} which is the result of the \textbf{ocamlmktop}
command.

\subsection{Making class names, field signatures and method signatures}

Imagine you want to access the method
\textbf{m:(Ljava.lang.String;)V} and the field \textbf{f:I} of the
class \textbf{A}.

You first need to build the signatures associated to each entity.
According to the \emph{Javalib} API you will write:

\begin{verbatim}
    open Javalib_pack
    open JBasics
    let aname = make_cn "A"
    let java_lang_string = make_cn "java.lang.String"
    let ms =
    make_ms "m" [TObject (TClass java_lang_string)] None
    let fs = make_fs "f" (TBasic `Int)
\end{verbatim}
\subsection{Getting a class representation from a binary file}

The methods you need are in the \emph{Javalib} module. You can open
this module cause you will need it very often.

\begin{verbatim}
    open Javalib
\end{verbatim}
Then, you need to build a \textbf{class\_path} to specify where the
classes you want to load have to be found:

\begin{verbatim}
    let class_path = class_path "./" (* for instance *)
\end{verbatim}
You can now load the class \textbf{./A.class} corresponding to
\textbf{aname}.

\begin{verbatim}
    let a = get_class class_path aname
\end{verbatim}
When you don't need a classpath any more, close it with
\textbf{close\_class\_path} if you don't want to get file
descriptors exceptions in a near futur.

\subsection{Getting fields and methods from a class}

You now have the class \textbf{a} of type
\emph{Javalib.interface\_or\_class}. You might want to recover its
method \textbf{m} of type \emph{Javalib.jmethod} and field
\textbf{f} of type \emph{Javalib.any\_field}.

Simply do:

\begin{verbatim}
    let m = get_method a ms
    let f = get_field a fs
\end{verbatim}
\begin{description}
\item[Note:]
The methods \textbf{get\_method} and \textbf{get\_field} raise the
exception \textbf{Not\_found} if the method or field asked for
can't be found.
\end{description}
It's important to notice that \textbf{a} can be a
\emph{Class of jclass} or an \emph{Interface of jinterface} (see
type \emph{interface\_or\_class}), but that the methods
\textbf{get\_method} and \textbf{get\_field} work equaly on it.
That's why \textbf{get\_field} returns a value of type
\emph{any\_field} which can be \emph{ClassField of class\_field} or
\emph{Interface\_field of interface\_field}. Indeed, according to
the JVM specification, we need to make the distinction between
interface fields and class fields.

\subsection{A more sophisticated example}

Now we would like to write a function that takes a
\textbf{classpath} and a \textbf{classname} as parameters and that
returns, for each method of this class, a set of the fields
accessed for reading (instructions \textbf{getstatic} and
\textbf{getfield}).

Here is the code:

\begin{verbatim}
    open Javalib_pack
    open Javalib
    open JBasics
    open JCode

    let get_accessed_fields (class_path : class_path)
     (cn : class_name) =
     (* We first recover the interface or class associated to the
        class name cn. *)
     let c = get_class class_path cn in
     (* Then, we get all the methods of c. *)
     let methods : jcode jmethod MethodMap.t = get_methods c in
     (* For each method of c, we associate a field set containing
        all the accessed fields. *)
       MethodMap.map
        (fun m ->
          match m with
          (* A method can be abstract or concrete. *)
           | AbstractMethod _ ->
             (* An abstract method has no code to parse. *)
              FieldSet.empty
           | ConcreteMethod cm ->
              (match cm.cm_implementation with
              (* A concrete method can be native so that we don't
                 know its behaviour. In this case we suppose that
                 no fields have been accessed which is not safe. *)
                | Native -> FieldSet.empty
                | Java code ->
                  (* The code is stored in a lazy structure, for
                     performance purposes. Indeed when loading a
                     class the Javalib does not parse its methods. *)
                   let jcode = Lazy.force code in
                   (* We iter on the array of opcodes, building our
                      field set at the same time. *)
                     Array.fold_left
                      (fun s op ->
                        match op with
                         | OpGetField (_, fs)
                         | OpGetStatic (_, fs) ->
                      (* We add the field signature in our field set.
                         In this example, we ignore the classes in
                         which the fields are defined. *)
                            FieldSet.add fs s
                         | _ -> s
                      ) FieldSet.empty jcode.c_code
              )
        ) methods
\end{verbatim}
This method has the signature

\begin{verbatim}
    Javalib.class_path ->
      JBasics.class_name -> JBasics.FieldSet.t JBasics.MethodMap.t
\end{verbatim}
\subsection{Another use case}

Consider the following class written in java:

\begin{verbatim}
    public class TestString{
       public boolean m(String s){
          if (s.equals("str")){
             return true;
          } else{
             return false;
          }
       }
    }
\end{verbatim}
We see that the method \emph{m} might raise an \emph{NullPointer}
exception if we call the method \emph{equals} on an uninitialized
string \emph{s}. To avoid this, a good practice is to replace the
test \textbf{s.equals(``str'')} by the expression
\textbf{``str''.equals(s)} which will return false rather than
raising an exception.

Let's see the bytecode associated to the method \emph{m}, given by
\textbf{javap}:

\begin{verbatim}
    public boolean m(java.lang.String);
      Code:
       0:   aload_1
       1:   ldc     #2; //String str
       3:   invokevirtual   #3; //Method
               java/lang/String.equals:(Ljava/lang/Object;)Z
\end{verbatim}
We will now write a sample of code that detects instructions of
type \textbf{ldc `string'} followed by an \textbf{invokevirtual} on
\emph{java.lang.String.equals} method.

We first need to write a function that returns the next instruction
and its program point in a code, given this code and a current
program point:

\begin{verbatim}
    let rec next_instruction (code : jopcodes) (pp : int)
      : (jopcode * int) option =
     try
       match code.(pp+1) with
        | OpInvalid -> next_instruction code (pp+1)
        | op -> Some (op,pp+1)
     with _ -> None
\end{verbatim}
Now we define a function that takes a \emph{classpath} and a
\emph{classname} as parameters and that returns a map associating
each concrete method signature to a list of
(\textbf{int},\textbf{string}) couples representing the program
points and the strings on which the \emph{java.lang.String.equals}
method is called.

\begin{verbatim}
    let get_equals_calls (class_path : class_path)
      (cn : class_name) =
     (* We first recover the interface or class associated to the
        class name cn. *)
     let java_lang_string = make_cn "java.lang.String" in
     let equals_ms =
       make_ms "equals" [TObject (TClass java_lang_object)]
        (Some (TBasic `Bool)) in
     let c = get_class class_path cn in
     (* Then, we get all the concrete methods of c. *)
     let methods : jcode concrete_method MethodMap.t =
       get_concrete_methods c in
     (* For each concrete method of c, we associate a (int*string) list
        containing all the strings passed as parameters to
        String.equals method, associated to the program point where the
        call occurs. *)
       MethodMap.map
        (fun m ->
         (match m.cm_implementation with
           (* A concrete method can be native so that we don't
              know its behaviour. In this case we suppose that
              no call to String.equals which is not safe. *)
           | Native -> []
           | Java code ->
             (* The code is stored in a lazy structure, for
                performance purposes. Indeed when loading a
                class the Javalib does not parse its methods. *)
              let jcode = Lazy.force code in
              let code = jcode.c_code in
              let l = ref [] in
              (* We iter on the array of opcodes, building
                  our list of (int*string) at the same time. *)
                Array.iteri
                 (fun pp op ->
                   match op with
                    | OpConst (`String s) ->
                      (* We detect that a string s is pushed on the
                         stack. The next instruction might be an
                         invokevirtual on String.equals. *)
                       (match (next_instruction code pp) with
                         | Some (inst,ppi) ->
                            (match inst with
                              | OpInvoke (`Virtual (TClass cn), ms)
                                 when cn = java_lang_string
                                   && ms = equals_ms ->
                                (* We add the program point of the
                                   invokevirtual and the pushed string
                                   in our list. *)
                                 l := (ppi, s) :: !l
                              | _ -> ()
                            )
                         | None -> ()
                       )
                    | _ -> ()
                 ) code;
              (* We simply return our list, in the reverse order so that
                 the program points appear in ascending order. *)
                List.rev !l
         )
        ) methods
\end{verbatim}
This method has the signature

\begin{verbatim}
    Javalib.class_path ->
      JBasics.class_name -> (int * string) list JBasics.MethodMap.t
\end{verbatim}
We obtain the expected result on the previous class
\emph{TestString}:

\begin{verbatim}
    # let cp = class_path ".";;
    val cp : Javalib.class_path = <abstr>

    # let cn = make_cn "TestString";;
    val cn : JBasics.class_name = <abstr>

    # let mmap = get_equals_calls cp cn;;
    val mmap : (int * string) list JBasics.MethodMap.t = <abstr>

    # let l = 
        let sk = List.map (ms_name) (MethodMap.key_elements mmap)
        and sv = MethodMap.value_elements mmap in
          List.combine sk sv;;
    val l : (string * (int * string) list) list =
      [("m", [(3, "str")]); ("<init>", [])]

    # let () = close_class_path cp;;
\end{verbatim}

\end{document}
